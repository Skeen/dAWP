The Fun Sorting game is implemented as a table with 10 columns, each containing a draggable span.
Each span has a unique id, via. the naming scheme $drag(i) \:|\: i \in \{1,10\}$.
The spans are sized and colored via. a small css snippet in 'ex3.html'.
\newline
The spans are hooked up to javascript via. the events; 'ondragstart', 'ondrop' and 'ondragover':
\begin{minted}[linenos, frame=lines]{html}
  <span id="drag1" draggable="true" ondragstart="drag(event)"
   ondrop="drop(event)" ondragover="allowDrop(event)"></span>
\end{minted}
The game is started via. the 'onload' event of the html body. 
\begin{minted}[linenos, frame=lines]{html}
  <body onload="init()">
\end{minted}
Once the html page is loaded, control is passed to the javascript function 'init', via. the onload event of the html body.
The init function calls the reset function, which runs through all 10 spans, generating and adding a random number for each of them.
\newline
It also checks if 'Storage' is supported by the browser, and if it is uses it to render the number of games played.
If it is not supported, an error is written to notify the player.

At this point the initial game is setup for playthrough.
\newline\newline
The drag and drop handles are quite straight forward. When an item is dragged (picked up), it's id is saved for when it's dropped.
When the item is dropped, we check were we're dropped, and exchange to dragged and the one we're dropped on.
We then check if the game is over (i.e. correctly sorted), if this is the case, we bump up the games played counter, and reset for another game.

\begin{minted}[linenos, frame=lines]{js}
function reset()
{
    var numberRow = document.getElementById('numberRow');
    // Loop through all the spans
    for(i = 0; i < numberRow.children.length; i++)
    {
        // Generate a random number between 1-10000
        var value = Math.floor((Math.random() * 10000) + 1);
        // Set the span's innerHTMl to this number
        document.getElementById("drag"+(i+1)).innerHTML = value;
    }
}
\end{minted}
