\documentclass[a4paper,10pt]{article}
\usepackage[utf8]{inputenc}
\usepackage{listings}
\usepackage{minted}

\author{
Søren Krogh -  20105661 \\
Emil Madsen - 20105376  \\
K. Rohde Fischer - 20052356\\}
\title{}
\begin{document}
\maketitle

\section*{Introduction}
To assist in understanding the project, here's a short introduction to
the various parts of it.  The unit testing is a part of the multiset.
Throughout the documents the JavaScript is loaded as the last thing.
This structure is becoming used in modern web development, to ensure
that visuals load before the JavaScript gets parsed, so the user gets
fast visual feedback.  Here it's just used because that structure is
slowly beginning to be considered as a good coding standard.

\section*{Unit testing}
A simple unit testing framework is created to make unit testing
easier.  The structure is fairly simple and is roughly based on a talk
by Christian Johansen.

The testing framework works by iterating an object and running all
functions inside it.  Two special function names are defined, $init$
and $finalize$, which are used to prepare each test and finalize it if
necessary.  Otherwise these are ignored.

As the way to loop all inner objects of an object is by using the for
... in ... construction there are multiple ways to prevent the
execution of those as tests.  In this tester they are just ignored by
matching the names $init$ and $finalize$ and just skipping those.
Another way would have been to use the DontEnum-construction of
JavaScript.  This is not used to two reasons: then it works by a
naming convention rather than a special construction (even if we want
that the naming convention is needed for JavaScript to find those),
and the construction of DontEnums are rather complex:

\begin{minted}[linenos, frame=lines]{js}
  // The test object containing tests
  var testObject = {
    test1: function() { /* ... */ },
    test2: function() { /* ... */ }
    // ...
  }

  Object.defineProperty(testObject, "init", { dontenum: true })
\end{minted}

This can also be done by overriding the $propertyIsEnumerable$
function or by introducing inheritance through prototypes.

It is noteworthy that the testing system could be improved by using
the exception structure when a test failes, because that would allow
multiple assertions inside each test function.  Another noteworthy
thing is that due to the for ... in ... construction and the way
JavaScript just handly object names as strings unit tests could be
developed like this:

\begin{minted}[linenos, frame=lines]{js}
  // The test object containing tests
  var testObject = {
    "should do something": function() { /* ... */ },
    "should do something different": function() { /* ... */ }
    // ...
  }
\end{minted}

To enable usage of natural language and let us simply use the function
names to provide good output when a test fails.  This has not been
used here though.

\section*{Multiset}
The multiset is created using test-driven development.  Most of the
implementation is quite straight forward (for details on the testing
see the unit testing section).  There was some uncertainty on whether
it should use $toString$ or $hashCode$ for the keys.  The design
chosen was $toString$ due to simplicity.  This ensures that all
existing JavaScript objects works in the selected implementation
without creating custom objects.  If $hasCode$ was used we would have
had to create custom objects for the purpose of testing.

The output from $toString$ on the multiset is unspecified in the
documentation.  Since a set is independent of the order of objects it
will be very hard to do proper testing without using $RegEx$.  For
these reasons we have chosen a more web friendly output for the
toString that wraps the items on separate lines and adds a class to
the key, enabling CSS styling.  The output is thus also tested by
visual verification rather than unit testing.

\section*{Ice cream shop}
The Ice Cream Shop follows a simple pattern where it listens for
changes and ensure that the given constraints are followed.  The design is
kept as simple as possible.

\section*{Klotski}
The Klotski game works by selecting a tile and then using the arrow
keys to move it around.  To control the keys an event listener waits
for key presses, then it validates the move, to see if it is legal and
then performs the move.

At a few places variables are used to store in between calculations
due to the double loop structure.  Problems can occur if the move is
performed during the loop, since the loop might continue to the new
location of a tile and re-perform the move.

\end{document}
