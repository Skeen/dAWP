%%% TeX-command-extra-options: "-shell-escape"
\documentclass[a4paper,10pt]{article}
\usepackage[utf8]{inputenc}
\usepackage{listings}
\usepackage{minted}

\author{
Søren Krogh -  20105661 \\
Emil Madsen - 20105376  \\
K. Rohde Fischer - 20052356\\}
\title{}
\begin{document}
\maketitle

\section*{Introduction}

\section*{Unit testing}
A simple unit testing framework is created to make unit testing
easier.  The structure is fairly simple and is roughly based on a talk
by Christian Johansen (unfortunately it is no longer available on the
net).

The testing framework works by iterating a test object and running all
functions inside it.  Two special function names are defined, $init$
and $finalize$, which are used to prepare each test and finalize it if
necessary.  Otherwise these are ignored.

As the way to loop all inner objects of an object is by using the for
... in ... construction there are multiple ways to prevent the
execution of those as tests.  In this tester they are just ignored by
matching the names $init$ and $finalize$ and just skipping those.
Another way would have been to use the DontEnum-construction of
JavaScript.  This is not used to two reasons: then it works by a
naming convention rather than a special construction (even if we want
that the naming convention is needed for JavaScript to find those),
and the construction of DontEnums are rather complex:

\begin{minted}[linenos, frame=lines]{js}
  // The test object containing tests
  var testObject = {
    test1: function() { /* ... */ },
    test2: function() { /* ... */ }
    // ...
  }

  Object.defineProperty(testObject, "init", { dontenum: true })
\end{minted}

This can also be done by overriding the $propertyIsEnumerable$
function or by introducing inheritance through prototypes.

It is noteworthy that the testing system could be improved by using
the exception structure when a test failes, because that would allow
multiple assertions inside each test function.  Another noteworthy
thing is that due to the for ... in ... construction and the way
JavaScript just handly object names as strings unit tests could be
developed like this:

\begin{minted}[linenos, frame=lines]{js}
  // The test object containing tests
  var testObject = {
    "should do something": function() { /* ... */ },
    "should do something different": function() { /* ... */ }
    // ...
  }
\end{minted}

To enable usage of natural language and let us simply use the function
names to provide good output when a test fails.  This has not been
used here though.

\section*{Multiset}

\section*{Ice cream shop}

\section*{Klotski}

\section*{Conlusion}

\end{document}
