% Write a 5 page tutorial about TypeScript, explaining
% 1. the new language features (target audience: someone familiar with JavaScript but not TypeScript),
% 2. how the most interesting new language features are compiled to JavaScript (with your thoughts on how the resulting code works),
% 3. how the type checking works, and
% 4. pros and cons compared to JavaScript and Dart.

\subsection*{New language features}
\subsubsection*{Type annotations}
TypeScript enchances JavaScript with the addition of optional type
annotations.  The type annotations are compile-time only, in order to
make it easier to make static analysis tools ti guide JavaScript
development.  That is, the types are not, in any way, present at
runtime. They do however allow the programmer to state the intent of
types in the program.

Say we wanted to create a function which adds two numbers;
\begin{minted}[linenos, frame=lines]{js}
function add(n1, n2) {
    return n1 + n2;
}
add(1, 5);
\end{minted}
Nothing wierd this far, however contrary to what may be expected, we can do this invokation as well;
\begin{minted}[linenos, frame=lines]{js}
add("a", "b");
\end{minted}
In this tiny example, there doesn't seem to be an issue, as the code is indeed still valid.

Say we wrote a unit test to the above code alike this;
\begin{minted}[linenos, frame=lines]{js}
function comm_test_add(n1, n2) {
    assert(add(n1,n2) === add(n2,n1))
}
comm_test_add(1,2);
\end{minted}

We'll now see that our commutativity property of plus fails, when the
add function is given string arguments.  As we should never break the
mathmatical properties of operators, because this would violate their
definitions and what can be proven from the definitions.  We'd like to
limit our add function to only accept types, for which the operator
properties hold.  The way to do this in TypeScript would be to hookup
type annotations for our add function;

\begin{minted}[linenos, frame=lines]{ts}
function add(n1 : number, n2 : number) {
    return n1 + n2;
}
\end{minted}
Now trying to do our previous invokation;
\begin{minted}[linenos, frame=lines]{js}
add("a", "b");
\end{minted}
Does not go through, but is instead statically captured by the TypeScript compiler, which produces the following error message;
\begin{minted}[linenos, frame=lines]{js}
add.ts(4,17): error TS2345: Argument of type 'string' is not
    assignable to parameter of type 'number'
\end{minted}
We've now ensured that the poor mathmatician doesn't have to cry over the broken properties of addition.

\subsubsection*{Structured data interfaces}
When writing methods that accept JavaScript Objects as arguments,
we usually have a fairly strict idea of which fields to expect on the object
(We should at least).

However JavaScript does not directly support a language feature to express this; Introducing data interfaces:
\begin{minted}[linenos, frame=lines]{ts}
interface Person {
    firstname: string;
    lastname: string;
}
\end{minted}
This is essence introduces a new type, for us to use as an optional type annotation.
It allows for the compiler to statically check for the excistance of the required fields on provided objects.
\\
Trying to call a function, without the required arguments yields a compiler error;
\begin{minted}[linenos, frame=lines]{ts}
function greeter(person : Person) {
    return "Hello, " + person.firstname + " " + person.lastname;
}
greeter({firstname: "Peter"});

------
greeter.ts(10,9): error TS2345:
    Argument of type '{ firstname: string; }' is not assignable
        to parameter of type 'Person'.
  Property 'lastname' is missing in type '{ firstname: string; }'.
\end{minted}

\subsubsection*{Classes}
TypeScript introduces the well-known concept of classes, complete with the concept of constructors, member methods, inheritance, alike.
A thorough examination of this language feature will not be executed here, we will instead focus on the compilation of TypeScript classes to JavaScript.

\subsection*{TypeScript to JavaScript compilation}
When compiling TypeScript classes to JavaScript, there is actually some transformation going on, rather than just removal of code, as is the case with the optional type annotations.

Below is an example of a simple TypeScript class implementation;
\begin{minted}[linenos, frame=lines]{ts}
class Greeter {
    greeting: string;
    constructor(message: string) {
        this.greeting = message;
    }
    greet() {
        return "Hello, " + this.greeting;
    }
}
\end{minted}
The code is very self explainatory, and hence we'll just move on to the compiled JavaScript output;
\begin{minted}[linenos, frame=lines]{js}
var Greeter = (function () {
    function Greeter(message) {
        this.greeting = message;
    }
    Greeter.prototype.greet = function () {
        return "Hello, " + this.greeting;
    };
    return Greeter;
})();
\end{minted}
We see that the class has now been replaced by a JavaScript object.
The constructor has been created as an explicit function, and everything is as-if we've written the class ourselves in JavaScript.
That is, the JavaScript compilation output resembles the JavaScript class idiom.
It's worth nothing that the greet method is now placed on the prototype, as it ought to be.

\subsubsection*{Inheritance}
Things get a little bit interesting when we inspect inheritance, in regards to the compiler output, given the below TypeScript class hierarchy;
\begin{minted}[linenos, frame=lines]{ts}
class Animal {
    constructor(public name: string) { }
    move() { alert(this.name + " moved "); }
}
class Snake extends Animal {
    constructor(name: string) { super(name); }
    move() { super.move(); }
}
\end{minted}
We do get somewhat more code from the compiler, than went in;
\begin{minted}[linenos, frame=lines]{ts}
var __extends = this.__extends || function (d, b) {
    for (var p in b) if (b.hasOwnProperty(p)) d[p] = b[p];
    function __() { this.constructor = d; }
    __.prototype = b.prototype;
    d.prototype = new __();
};
var Animal = (function () {
    function Animal(name) { this.name = name; }
    Animal.prototype.move = function ()
    { alert(this.name + " moved "); };
    return Animal;
})();
var Snake = (function (_super) {
    __extends(Snake, _super);
    function Snake(name) { _super.call(this, name); }
    Snake.prototype.move = function ()
    { _super.prototype.move.call(this); };
    return Snake;
})(Animal);
\end{minted}
First, we have a piece of code, which implements the inheritence pattern (lines $1\rightarrow6$).
This is automatically injected by the compiler, whenever inheritance is found within the TypeScript code.
\\
The generated code for Animal, is as we'd expect i.e. in regards to the Greeter class in the previous section.
\\
The snake class is where it gets interresting. We have an injected call to the injected inheritance pattern code, which ensures that we get to inherit properties correctly.
The rest of the code is somewhat straight forward.

\subsection*{Comparison versus JavaScript and Dart}
The languages are very different in regards to type-safety and static analysis tools.
Ranging from not providing anything at all (JavaScript), to ad-hoc post fitting a type system (TypeScript), all the way to designing a language with a fully fledged type system (Dart).
\\
We generally want as much static analysis, and help as feasible, and hence the question of feasibility becomes central.
Feasibility is however, a very subject being, both in terms of the developer and what's being developed.
\\
While one developer may find it feasible to learn an entire new language, toolchain and framework for interacting with JavaScript and/or the DOM.
\\
Other developers may just be going for the easier approach found in TypeScript, and settle with less help through static analysis.
\\
Some developers may even accept that they're all on their own, and just write pure JavaScript, wishing for the best of luck.

So generally speaking, it's just a question of the effort the developer wants to put into getting reasonable type annotations and static analysis.
Both TypeScript and Dart compiles down to JavaScript, and hence can be run on the same platforms.
Debugging the compiled JavaScript from the Dart compiler, is however something of a challenge, compared to the TypeScript's JavaScript output, simply because while TypeScript remains a somewhat 1-1 correspondance, Dart does not.

We do believe that Dart is a very interresting language in regards to getting the typesafety and static analysis that we want, however TypeScript is just a much easier choice, as it's just a superset of JavaScript.
