% Create an AngularJS directive that can be used as follows to show a number of stars (like in the homework peer-reviewing system):
% <rating stars="3" max="5" />
% Bonus: make it possible to adjust the rating by clicking on a star, and such that a callback is triggered whenever the user changes the rating (allowing, for instance, the client code to send the changes to the server).
Note: We did not do the bonus part of this exercise.

We've created an \verb|index.html| file, with the following contents;
\begin{minted}[linenos, frame=lines]{html}
...
<body ng-app="magica">
<div ng-controller="Controller">
    <rating stars="4" max="5"/>
</div>
</body>
...
\end{minted}
Where we're utilizing our rating directive, which is defined within our \verb|script.js|;
\begin{minted}[linenos, frame=lines]{js}
controller.directive('rating', function()
{
    return { template: function(elem, attr) {
        var stars = attr['stars'];
        var max = attr['max'];
        if((stars > max) || (stars < 0)) {
            return "Invalid number of stars";
        }
        var star_html = '<img src="star.jpg" height="42" width="42"/>';
        function repeat(str, times) {
            return new Array(times + 1).join(str);
        }
        return repeat(star_html, Number(stars));
    }};
});
\end{minted}
We consider the code very self-explainatory, but we've like to add a single remark.
One could argue that we've should use a more AngularJS stylish approach to generating the number of stars.
We've decided to go with the poor-mans approach of simply duplicating an image tag,
as the focus on of the exercise was to do the directive, not the implementation details of what it does.
