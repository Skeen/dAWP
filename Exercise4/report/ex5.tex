Below is the 5 language features of Dart, that are unlike Java, but
are interresting to us.

\begin{itemize}
\item Optional type annotations: When writing small programs, scripts
  alike, it's often abit overkill to actually write out all the type
  annotations, simply because the complexity of the entire program is
  small enough, that the type annotations do not lift their own weight
  in usefulness.  Java has a known issue about the amount of
  boiler-plate code being massive, and forced type annotations is a
  part of that.

  To prove the point, below is the minimal hello world program in both
  Java and Dart.
  \begin{minted}[linenos, frame=lines]{dart}
    void main()
    {
      print("hello world");
    }
  \end{minted}
  And the Java equivalent;
  \begin{minted}[linenos, frame=lines]{java}
    public class HelloWorld
    {
      public static void main(String[] args)
      {
        System.out.println("Hello, World");
      }
    }
  \end{minted}

\item Top-level functions: This applies to the rest of the
  boiler-plate code of the above program, namely that in Java every
  single function will have to be wrapped in a class.  In Dart
  however, we can have functions defined on the top-level
  (i.e. outside of classes). Which shortens small programs, and makes
  code more readible.  And alternative to this, is the use of
  \href{http://projectlombok.org}{Project Lambok} in Java, which uses
  annotations to auto-generate all of Javas boiler-plate code.

\item Operator Overloading: While operator overloading is usually a
  discussed features of programming languages, it does have it's
  applications. - That is; when it's not misused.  Below is an example
  of a bit of matrix code, first with operator overloading, then
  without:
  \begin{minted}[linenos, frame=lines]{c++}
    E  =  A * (B / 2);
    E += (A - B) * (C + D);
    F  =  E;                  // deep copy of the matrix
  \end{minted}
  The above is valid C++, using operator overloading, all variables
  are matricies. Below is the equivalent implemented in Java:
  \begin{minted}[linenos, frame=lines]{c++}
    E = A.times(B.divide(2));
    E = E.plus(A.minus(B).times(C.plus(D)));
    F = E.copy();             // deep copy of the matrix
  \end{minted}
  Striving the achieve clarity and readability, it's obvious that the
  Java code fails, because of the immense verbosity and unwanted
  decoupling from mathmatical notation.  Operator overloading allows
  us to achieve the desired syntactical representation, which
  immitates usual mathmatical notation.

\item $Future$ is used to handle things that might happen in the
  future.  Because Dart is single threaded it can be quite problematic
  waiting for an I/O operation or remote requests.  To handle this
  $Future$ is basically creating a asynchronious callback.  If a
  similar setup was to be done in a language such as Java threaded
  operations will be necessary to prevent blocking.

  $Futures$ also has chained calls to make it easier to handle return
  values from the $Future$, which for instance is useful when reading
  a file as the content will be exactly this return value.  There's
  also chains for error handling delayed actions.  All of which would
  require different more or less complex inheritances and threading.

\item For handling events Dart uses the future system for the event
  loop.  As Dart is single threaded there's no guarantee of when an
  event will be executed.  Ie. if a long running task blocks execution
  a timer event will not occur till after the long running task has
  ended, even if the time is up.  In languages such as Java the event
  loop will usually be part of the GUI toolkit that you are using and
  it will be running in a separate thread.  This prevents most
  blocking cases but in turn makes the program structure harder to
  analyze and it can be the cause for bugs as updating the GUI from a
  different thread than the GUI's own isn't always allowed.

  In the event loop structure of Dart there is a microtask queue that
  is always executed before the next event on the event queue.  This
  is used for controlling the order of tasks when an event use the
  chained then and for different small tasks.  In turn though this can
  cause starvation of the event loop if long running tasks is run as
  microtasks or if more microtasks keeps being added.

  % Idea; compilation to javascript, versus java virtual machine / applet.

\end{itemize}

