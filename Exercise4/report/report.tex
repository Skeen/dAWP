\documentclass[a4paper,10pt]{article}
\usepackage[utf8]{inputenc}
\usepackage{listings}
\usepackage{minted}
\usepackage{hyperref}

\author{
Søren Krogh -  20105661 \\
Emil Madsen - 20105376  \\
K. Rohde Fischer - 20052356\\}
\title{Exercises 4 - Dart and Flapjax}
\begin{document}
\maketitle

\section*{Exercise 1}
Trivially solved, no issues.

\section*{Exercise 2}
% Re-implement the multiset datatype from week 1 using TypeScript. Make sure to take advantage of the new features in TypeScript.
The multiset in TypeScript is fairly easy.  One thing that could have
been done to make it resemble the Java version more is to use the
generics from TypeScript, which hasn't been done here.  Apart from
that it's quite straight forward using classes.

Two noteworthy things is that using classes makes the one-shot
closures becomes less natural (although still possible) for
encapsulating $trueRemove$ and the internal data types and that
TypeScript does not like the overload with two different return types.
This would probably be best handled by dividing it up in two functions
with different names.

\section*{Exercise 3}
\begin{minted}[linenos, frame=lines]{js}
main() {
  print(plus(1,2));
}

plus(String s1, String s2) {
  return s1 + s2;
}
\end{minted}
The first program prints the result of the function 'plus' given two parameters. The function 'plus' requires two string parameters, but in this program, it is called with two int parameters. When the program is run in 'checked mode' we will get a type error because we try to call a function with the wrong type parameters. When the program is run in 'production mode' the program runs and prints '3'. This happens because the it ignores the string type of the parameter of the 'plus' function, and perform the plus on the ints 1 and 2 which results in 3.

\begin{minted}[linenos, frame=lines]{js}
main() {
  Object o1 = 1;
  Object o2 = 2;
  print(plus(o1,o2));
}

plus(String s1, String s2) {
  return s1 + s2;
}
\end{minted}
The second program is much the same as the first program. In this program though two objects o1 and o2 is initialized to 1 and 2 respectively in the main function. If the program is run in 'checked mode' as before we get a type error because we try to pass two objects as parameters to a function which takes two strings. If the program is run in 'production mode' again the program runs and prints '3'. This is for the same reason as the first program.\\

When runnig both programs in 'cheked mode' we got type errors. This is what one would expect since we try to pase parameters of the wrong type. In 'production mode' we got no warnings. Since 'checked mode' is used for developing and 'production mode' is used for launching the product it makes sence to get the warnings during development and the run with the "keep on truckin" philosophy after the product is launched.\\

Java is statically typed which means that all variables need to be declared before they can be used. This involves setting the variables type. The type determines the values a variable may contain and the operations that may be performed on it. Dart is dynamically typed and has optional types, which means that you can write programs with no types whatsoever. If you do choose to add types, this will not prevent the program from compiling and running, even if the types are incomplete or wrong.\\

\section*{Exercise 4}
The async keyword is added to the function declaration, and marks the
function as asynchronous, making it return a \verb|Future<T>|
immidately, rather than computing the function and returning \verb|T|
itself.  The function is scheduled to be executed, and runs
asynchronously to the caller.

The await keyword awaits the completion of computation for a
\verb|Future<T>|, in layman terms, it suspoens execution while waiting
for the future to complete.  The result of the await expression on a
\verb|Future<T>| is \verb|T|.

In short, async is used to run functions asynchronously, wrapping
their return types in futures. - This allows one to easily write
asynchronous code.  Await is used to wait for asynchronous functions
to compute, as if they were non-async. - This allows one to easily
write code which is asynchrous as if it was synchronous.

In essence the convoluted callback handling of Javascript is avoided
or the verbose use of '\verb|.then(T)|' and heavy use of lambdas from
Dart futures or the Javascript Q library. While retaining the
advantages of asynchronous evaluation:
\begin{minted}[linenos, frame=lines]{dart}
file.exists().then((bool exists) {
    if (!exists) {
      file.create(recursive: true).then((File file) {
        file.writeAsString("version=1");
      })
      .catchError(handleError);
    } else {
      file.readAsString().then((String text) {
        int version = int.parse(text.split("=").last);
        file.writeAsString("version=" + (version+1).toString());
      })
      .catchError(handleError);
    }
  });
\end{minted}
Can be rewritten to the below using async and await;
\begin{minted}[linenos, frame=lines]{dart}
if(!(await file.exists())) {
      try {
        (await file.create(recursive: true)).writeAsString("version=1");
      } catch(exception, stackTrace) {
        handleError(exception);
      }
  } else {
      try {
        int version = int.parse((await file.readAsString()).split("=").last);
        file.writeAsString("version=" + (version+1).toString());
      } catch(exception, stackTrace) {
        handleError(exception);
      }
  }
\end{minted}
As can be seen errors are now handled using the exception handling
mechanism, rather than using the catchError scheme from futures.

The same scheme as used in the second code snippet could be achieved
using the synchronous file IO system, however not gaining
the benefits of asynchronicity, if that is done.


\section*{Exercise 5}
Below is the 5 language features of Dart, that are unlike Java, but
are interresting to us.

\begin{itemize}
\item Optional type annotations: When writing small programs, scripts
  alike, it's often abit overkill to actually write out all the type
  annotations, simply because the complexity of the entire program is
  small enough, that the type annotations do not lift their own weight
  in usefulness.  Java has a known issue about the amount of
  boiler-plate code being massive, and forced type annotations is a
  part of that.

  To prove the point, below is the minimal hello world program in both
  Java and Dart.
  \begin{minted}[linenos, frame=lines]{dart}
    void main()
    {
      print("hello world");
    }
  \end{minted}
  And the Java equivalent;
  \begin{minted}[linenos, frame=lines]{java}
    public class HelloWorld
    {
      public static void main(String[] args)
      {
        System.out.println("Hello, World");
      }
    }
  \end{minted}

\item Top-level functions: This applies to the rest of the
  boiler-plate code of the above program, namely that in Java every
  single function will have to be wrapped in a class.  In Dart
  however, we can have functions defined on the top-level
  (i.e. outside of classes). Which shortens small programs, and makes
  code more readible.  And alternative to this, is the use of
  \href{http://projectlombok.org}{Project Lambok} in Java, which uses
  annotations to auto-generate all of Javas boiler-plate code.

\item Operator Overloading: While operator overloading is usually a
  discussed features of programming languages, it does have it's
  applications. - That is; when it's not misused.  Below is an example
  of a bit of matrix code, first with operator overloading, then
  without:
  \begin{minted}[linenos, frame=lines]{c++}
    E  =  A * (B / 2);
    E += (A - B) * (C + D);
    F  =  E;                  // deep copy of the matrix
  \end{minted}
  The above is valid C++, using operator overloading, all variables
  are matricies. Below is the equivalent implemented in Java:
  \begin{minted}[linenos, frame=lines]{c++}
    E = A.times(B.divide(2));
    E = E.plus(A.minus(B).times(C.plus(D)));
    F = E.copy();             // deep copy of the matrix
  \end{minted}
  Striving the achieve clarity and readability, it's obvious that the
  Java code fails, because of the immense verbosity and unwanted
  decoupling from mathmatical notation.  Operator overloading allows
  us to achieve the desired syntactical representation, which
  immitates usual mathmatical notation.

\item $Future$ is used to handle things that might happen in the
  future.  Because Dart is single threaded it can be quite problematic
  waiting for an I/O operation or remote requests.  To handle this
  $Future$ is basically creating a asynchronious callback.  If a
  similar setup was to be done in a language such as Java threaded
  operations will be necessary to prevent blocking.

  $Futures$ also has chained calls to make it easier to handle return
  values from the $Future$, which for instance is useful when reading
  a file as the content will be exactly this return value.  There's
  also chains for error handling delayed actions.  All of which would
  require different more or less complex inheritances and threading.

\item For handling events Dart uses the future system for the event
  loop.  As Dart is single threaded there's no guarantee of when an
  event will be executed.  Ie. if a long running task blocks execution
  a timer event will not occur till after the long running task has
  ended, even if the time is up.  In languages such as Java the event
  loop will usually be part of the GUI toolkit that you are using and
  it will be running in a separate thread.  This prevents most
  blocking cases but in turn makes the program structure harder to
  analyze and it can be the cause for bugs as updating the GUI from a
  different thread than the GUI's own isn't always allowed.

  In the event loop structure of Dart there is a microtask queue that
  is always executed before the next event on the event queue.  This
  is used for controlling the order of tasks when an event use the
  chained then and for different small tasks.  In turn though this can
  cause starvation of the event loop if long running tasks is run as
  microtasks or if more microtasks keeps being added.

  % Idea; compilation to javascript, versus java virtual machine / applet.

\end{itemize}



\section*{Exercise 6}

\end{document}
