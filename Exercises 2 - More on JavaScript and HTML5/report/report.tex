\documentclass[a4paper,10pt]{article}
\usepackage[utf8]{inputenc}
\usepackage{listings}
\usepackage{minted}

\author{
Søren Krogh -  20105661 \\
Emil Madsen - 20105376  \\
K. Rohde Fischer - 20052356\\}
\title{Exercise 2 - More JavaScript and HTML5}
\begin{document}
\maketitle

\section*{Embedded JavaScript}

\begin{minted}[linenos, frame=lines]{js}
  void(window.setInterval(function() {document.title = new Date}, 1000))
\end{minted}
Running this code in the browser, returns 'undefined', and changes the title of the page to the date and time. the time is updated every second.
In JavaScript the void operator allows expressions that produce side effects into places where an expression that evaluates to 'undifined' is desired. Hence 'undefined' is returned.

\begin{minted}[linenos, frame=lines]{js}
  foo = { bar: "CWP" }
  bar = "bar"
  foo.undefined = foo[bar]
  foo[foo.bar.baz]
\end{minted}
If this pice of code is run, "CWP" is returned.
First a new variable 'foo' is declared and a new object 'bar' with value "CWP" is constructed. Next 'bar' is set to be equal to the string "bar". Then 'foo.undefined' that otherwise would return 'undefined' is set to be equal to 'foo[bar]' which returns "CWP".
Because 'foo.undefined' is set to 'foo[bar]'executing 'foo[foo.bar.baz]' returns "CWP". 


\begin{minted}[linenos, frame=lines]{js}
  function Msg(m) {this.msg = m; this.number = ++Msg.prototype.counter;}
  Msg.prototype.counter = 0;
  Msg.prototype.show = function() {alert("Message number " + this.number + " is: " + this.msg);};
  var x = new Msg("hello");
  var y = new Msg("world");
  x.show();
  y.show();
  x.counter;
\end{minted}
This will show two dialog boxes, the first saying "Message number 1 is:hello" and the second "Message number 2 is:world" and then return 2.
First the function 'Msg' which takes one parameter is created. The function sets 'this.msg' to the parameter, and sets 'this.number' to ++"the function counter".
Next the counter is set to 0.
Then the property 'show' is added to the 'Msg' objects. 'show' creates an alert message using 'this.number' to get the number of the message and 'this.msg' to get the actual message. The two messages 'hello' and 'world' is created and assigned to the two variables 'x' and 'y' respectively. Then the two messages are shown using the 'show()' property. Lastly 'x.counter' is executed which returnes the current state of the Msg counter, which at this state afte creating the two messages is 2 .


\begin{minted}[linenos, frame=lines]{js}
  function Person(n) {
    this.name = n || "???"
    Person.prototype.count++
  }
  Person.prototype.count = 0
  function Student(n,s) {
    this.base = Person
    this.base(n)
    delete this.base
    this.studentid = s
  }
  Student.prototype = new Person
  var x = new Student("Joe Average", "100026")
  x.count
\end{minted}
Runnig this will return the value 2.
First the function 'Person' is created. This takes one parameter and creates a Person object with the given parameter as the name and the increments the counter. If no parameter is given, 'this.name' is set to "???".
Then the counter is set to 0.
Next the function 'Student' is created. 'Student' takes two parameters. A new property called 'base' is created and set to the value of the 'Person' constructor. Then the 'base' method is called and the 'n' parameter is given as argument. Then 'this.base' is deleted. lastly 'this.studentid' is set to the second parameter.
Dynamic inheritance is set up by executing "Student.prototype = new Person". This also increments the 'Person' counter. Then a new 'Student' object is created and assigned to the variable 'x'. Because 'Studen' inherits from 'Person', the 'Person' counter is incremented so when finaly 'x.count' is called, the value 2 is returned.


\section*{Object oriented hierarchy}
The simple inheritance hierachy is done by using the built in
prototype mechanism in JavaScript.  To test this a div box with the
outputs from the different calls has been created, as this creates the
simplest test bench possible.

It is possible to ensure that Pet can never be instantiated through
the one-shot closure:

\begin{minted}[linenos, frame=lines]{js}
  // Global scope variables
  // In actually they should probably be in a module
  var Dog;
  var Cat;

  function() {
    // This is a variable limited to the function scope.
    var Pet = function() { /* ... */ };

    Dog = function() { /* ... */ };
    Cat = function() { /* ... */ };
    Dog.prototype = new Pet()
    /* ... other prototype actions ... */
    Cat.prototype = new Pet()
    /* ... other prototype actions ... */
  }();
\end{minted}

For concealing the name property the only option is to create a
separate getName function for each subclass, thus eliminating the
smart practicality with inheritance.  This has been demonstrated in
the Dog class, but would be quite impractical in the long run.  The
solution require the usage of ``DontDelete'' ($configurable: false$)
and ``Read-only'' ($writable: false$).

The technique for making sound read-only is the same as for getName,
albeit more practical as they aren't defined on the inheritance level.

\section*{Fun Sorting Game}
The Fun Sorting game is implemented as a table with 10 columns, each containing a draggable span.
Each span has a unique id, via. the naming scheme $drag(i) \:|\: i \in \{1,10\}$.
The spans are sized and colored via. a small css snippet in 'ex3.html'.
\newline
The spans are hooked up to javascript via. the events; 'ondragstart', 'ondrop' and 'ondragover':
\begin{minted}[linenos, frame=lines]{html}
  <span id="drag1" draggable="true" ondragstart="drag(event)"
   ondrop="drop(event)" ondragover="allowDrop(event)"></span>
\end{minted}
The game is started via. the 'onload' event of the html body. 
\begin{minted}[linenos, frame=lines]{html}
  <body onload="init()">
\end{minted}
Once the html page is loaded, control is passed to the javascript function 'init', via. the onload event of the html body.
The init function calls the reset function, which runs through all 10 spans, generating and adding a random number for each of them.
\newline
It also checks if 'Storage' is supported by the browser, and if it is uses it to render the number of games played.
If it is not supported, an error is written to notify the player.

At this point the initial game is setup for playthrough.
\newline\newline
The drag and drop handles are quite straight forward. When an item is dragged (picked up), it's id is saved for when it's dropped.
When the item is dropped, we check were we're dropped, and exchange to dragged and the one we're dropped on.
We then check if the game is over (i.e. correctly sorted), if this is the case, we bump up the games played counter, and reset for another game.

\begin{minted}[linenos, frame=lines]{js}
function reset()
{
    var numberRow = document.getElementById('numberRow');
    // Loop through all the spans
    for(i = 0; i < numberRow.children.length; i++)
    {
        // Generate a random number between 1-10000
        var value = Math.floor((Math.random() * 10000) + 1);
        // Set the span's innerHTMl to this number
        document.getElementById("drag"+(i+1)).innerHTML = value;
    }
}
\end{minted}


\section*{Asynchronous loading}
\subsection*{XMLHttpRequest}
  XMLHttpRequrest(xhr) provides some security features e.g:
Same origin policy (sikkerhedspolitik). This tries to prevent cross domain attack and does not blindly parse code(You can still explicitly parse it). Thus it's still possible to implement the security errors.
Furthermore using xhr also gives acess to http statuses which can be used to provide better feedback.
A downside to xhr is that it can be rather tedious to code, but most libraries provide nice wrappers and if not using a library, it's fairly easy to code yourself.

\subsection*{iframe \& Script Tags}
Iframe and Script tags both have the advantage that they are rather light on code. On the other hand, neither iframe nor Script tags prevents cross origin attacks or execution of code. Usually this method is definitely not recomended but is in some cases used when browser permossions or similar prevents xhr. A common format in that case is JSONP.

\section*{Boss snooping}


\section*{JSLint}
The concept of validating JavaScript is justified because the interpretors operate by the "keep on trucking" philosophy. This in turn means that validation is limited at best. Thus JavaScript code is quite error prone and alot of pages have errors due to this fact.

The validation by JSLint is probably quite usefull, but seems to overly pedantic. For instance it require a whitespace in places where it has, and can never have, any semantic difference. 


\end{document}
