\documentclass[a4paper,10pt]{article}
\usepackage[utf8]{inputenc}
\usepackage{listings}
\usepackage{minted}

\author{
Søren Krogh -  20105661 \\
Emil Madsen - 20105376  \\
K. Rohde Fischer - 20052356\\}
\title{Exercises 3 - Javascript libraries and Node.js}
\begin{document}
\maketitle

\section*{Embedded JavaScript}

\begin{minted}[linenos, frame=lines]{js}
  void(window.setInterval(function() {document.title = new Date}, 1000))
\end{minted}


\section*{Object oriented hierarchy}

\section*{Fun Sorting Game}

\section*{Performance comparison}
To test the performance of the two ways of accessing attributes, two
for loops were created - one for each way.  Both loops change the
style attribute twice (the background alternates between red and
blue), and reads them between each step.  Both get a new date object
before the loop and after the loop, enabling a comparison in time.
Both loops run 10.000 times.

Strangely enough the behavior when tested actually contradicts the
statement from the lecture.  It seems that Firefox is consistent with
the statement whereas Chrome and reconq switch a bit between
confirming and contradicting, although both seems to contradict almost
consistently and the confirmation of the statement is the exception.
Another thing noted is that Chrome seems to do some optimization
(cache of the function results perhaps?), because after some tests,
the set/getAttribute-way took around 8-10 ms for all 10.000 calls.

This is most likely due to a lot of optimization in Chrome, and it is
suspected that if more attributes had been included in the test the
picture would have been different.

\section*{AJAX chat}

\section*{Inheritance}

\end{document}
