\documentclass[a4paper,10pt]{article}
\usepackage[utf8]{inputenc}
\usepackage{listings}
\usepackage{minted}

\author{
Søren Krogh -  20105661 \\
Emil Madsen - 20105376  \\
K. Rohde Fischer - 20052356\\}
\title{Exercises 3 - Javascript libraries and Node.js}
\begin{document}
\maketitle

\section*{Administrative System}
The json file is served locally as 'groups.js'.
The html file is somewhat minimalistic, in that we only have two selects and a single span.
\begin{itemize}
\item A select for the research groups (always shown)
\item A select for the research group member (shown when a research group is selected)
\item A span containing the research group administrator (shown when a research group is selected)
\end{itemize}
The research groups and group members select, both have a dummy nothing selected member.
As the research groups are only loaded once, theirs are hardcoded, while it's dynamical for the group members.
When the document is first loaded, we use jquery to download the json, and populate the research groups select.
\begin{minted}[linenos, frame=lines]{js}
$.getJSON("groups.json", function(json)
{
    // Add each research group to the drop down menu
    var research_group_selector = $('#research');
    $.each(data, function(key, value)
    {
        research_group_selector.append(
            $('<option></option>').val(key).html(value.name)
        );
    });
});
\end{minted}
Whenever a new research group is selected, we run a handler, which updates the group members select and administrator span accordingly. 
Assuming the dummy has been selected this handler will hide everything, if a research group is selected, their info will be loaded.
\begin{minted}[linenos, frame=lines]{js}
// Empty the member selector and add an empty option
var member_selector = $('#member');
member_selector.empty();
member_selector.append($('<option></option>'));
// Add each member as an option
$.each(research_table.members, function(key, value)
{
    member_selector.append(
        $('<option></option>').val(key).html(value)
    );
});

// Set the administrator
$('#administrator').text(research_table.admin);
\end{minted}

\section*{Drag-and-drop with jQuery UI}
For the reimplementation, jQuery UI's sortable was used.
The html was rewritten, so that the draggable elements is represented as li's in a ul in stead of a table.
The ul with id 'sortable', is set to be sortable, and after every update, checkVictory() is run to check if the list is correctly sorted.
The check is done by comparing the text of the current span-element with the next and checking if the first number is larger then the next. If so, the variable 'isSorted' which is initialiced to true before every check is set to false, and the game is not yet won.
If all the checks are done, and 'isSorted' is still true, the game is won, and an alert is displayed. Then the game is reset.

\section*{Performance comparison}
To test the performance of the two ways of accessing attributes, two
for loops were created - one for each way.  Both loops change the
style attribute twice (the background alternates between red and
blue), and reads them between each step.  Both get a new date object
before the loop and after the loop, enabling a comparison in time.
Both loops run 10.000 times.

Strangely enough the behavior when tested actually contradicts the
statement from the lecture.  It seems that Firefox is consistent with
the statement whereas Chrome and reconq switch a bit between
confirming and contradicting, although both seems to contradict almost
consistently and the confirmation of the statement is the exception.
Another thing noted is that Chrome seems to do some optimization
(cache of the function results perhaps?), because after some tests,
the set/getAttribute-way took around 8-10 ms for all 10.000 calls.

This is most likely due to a lot of optimization in Chrome, and it is
suspected that if more attributes had been included in the test the
picture would have been different.

\section*{Websocket chat}
We've decided to implement the chat client/server communication using websockets.

\subsection*{Client}
The client is somewhat minimalistic, in that we have 3 textareas, and a single button.
\begin{itemize}
\item A textarea for the username
\item A textarea for the chat history
\item A textarea for the writing messages
\item A button for sending the message
\end{itemize}

Upon loading the client, a connection to the server is made.
Once the connection is open, we send a greeting message to the server, in order to check connectivity.
When the user requests to send a message, we simply readout the textareas,
build up a json object, stringify it, and transfer it to the server.
\begin{minted}[linenos, frame=lines]{js}
submit.addEventListener("click", function(ev)
{
    var username = document.getElementById('username');
    var input = document.getElementById('input');

    var data = {
        sender:  username.value,
        message: input.value
    };

    connection.send(JSON.stringify(data));
    input.value = "";
});
\end{minted}
When a message is recieved from the server, we simply parse the json it sends,
and adds a formatted line to the chat history.
\begin{minted}[linenos, frame=lines]{js}
connection.onmessage = function(msg)
{
    var chat = document.getElementById('chat');
    var parsed = JSON.parse(msg.data);
    chat.value = chat.value + "\n" + parsed.sender + ": " + parsed.message;
};
\end{minted}

\subsection*{Server}
The server is written in node.js, and based upon the handed template.
We maintain a list of currently connected peers (we don't actually remove peers currently).
A peer is added to the list, each time it makes a connection to the server.
When a peer sends a valid message to the server, it's broadcasted to all peers.
Messages are valid, assuming their in utf8 encoding, json parsable and that the username is set.

\section*{Inheritance}
The inheritance is done using prototypejs, which made it fairly easy.
To make testing easier it has been made so the user can fill in a name
and press what kind of pet is desired.  Since the target audience here
is developers who intend to break the behavior, there is no validation
of the fields.  Thus tests such as accessing the empty property name
is possible.  The UI though does not encapsulate trying to instantiate
non-existing objects, but this can be tested through the JavaScript
console.

\end{document}
