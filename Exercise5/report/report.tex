\documentclass[a4paper,10pt]{article}
\usepackage[utf8]{inputenc}
\usepackage{listings}
\usepackage{minted}
\usepackage{hyperref}

\title{Vulnerabilities and Flapjax}
\author{Søren Krogh -  20105661 \\
Emil Madsen - 20105376  \\
K. Rohde Fischer - 20052356\\}
\begin{document}
\maketitle

\section*{Exploiting Gruyere}
\subsection*{Cross-Site Scripting}
The cross site scripting is a really set of attacks, because if
executed correctly.  The attacks can be used to achieve any number of
goals.  Most of them are based of taking completely control of the
site.  For instance the fact that file upload can be used to upload
websites enables us to set up a fake site that in all ways looks
authentic, because it by all means of validations is part of the
actual page.

Quite a few of the attacks uses the way data is inserted into the site
to inject JavaScript, for instance by exploiting the way attributes
are written to insert a harmful script.  For example if the color is
set to $\#"~onmouseover="alert(42)$ the tag for the user name will
have an onmouseover-attribute that can execute harmful code.

Also sending users URLs can enable execution of harmful code and also
the AJAX calls can be exploited because they execute the code returned
(that should be clean JSON, but might not be).

The XSS attacks provides a quite dangerous set of attacks as they
potentially provide full control over what the client does from the
second it is executed.  The attack can basically just add a script tag
providing the full attack code.

\subsection*{Client-State manipulation}
Manipulating the client stage provides is a way to forge requests, the
typical usage would be either escalating the users privileges such as
in the Gruyere example or for manipulating sites to have a wrong
behavior, such as a shop thinking the total price of a shopping basket
it 0.

\subsection*{Cross-Site Request Forgery}
The XSRF attack is a bit similar to the XSS, as it also provides a way
to make the user execute unintended actions.  The Gruyere example is
deleting a snippet, but could also be to make users upload intended
things.  

An interesting detail in the type of attack is that it can for
instance be used in cases where you don't have admin rights, but need
the admins permissions to do other attacks such as XSS.  Also due to
the fact the the simplest way to perform an XSRF is by the get
parameters, a lot of developers is wrongly let to believe post is more
secure.  This however is not true as the attacking site could have a
hidden form that submits to the site under attack and then having a
JavaScript that automatically submits.

\subsection*{Path Traversal}
Path traversal is used to either reveal secret info or to
place/replace files in the system.  This can be anything from
replacing a list of users to replacing for instance the index-page of
a site or even worse in the cases where the uploads are not properly
checked even uploading files that will be executed by the server.
This can in worst case be used to replace the entire code base of a
site with the code intended by a malicious person.

\section*{XSS and XSRF in node.js}

\section*{Flapjax}

\end{document}
