\documentclass[a4paper,10pt]{article}
\usepackage[utf8]{inputenc}
\usepackage{listings}
\usepackage{minted}
\usepackage{hyperref}
\usepackage{graphicx}

\title{Vulnerabilities and Flapjax}
\author{Søren Krogh -  20105661 \\
Emil Madsen - 20105376  \\
K. Rohde Fischer - 20052356\\}
\begin{document}
\maketitle

\section*{Exploiting Gruyere}
\subsection*{Cross-Site Scripting}
The cross site scripting is really a set of attacks, because if
executed correctly.  The attacks can be used to achieve any number of
goals.  Most of them are based of taking completely control of the
site.  For instance the fact that file upload can be used to upload
websites enables us to set up a fake site that in all ways looks
authentic, because it by all means of validations is part of the
actual page.

Quite a few of the attacks uses the way data is inserted into the site
to inject JavaScript, for instance by exploiting the way attributes
are written to insert a harmful script.  For example if the color is
set to $\#"~onmouseover="alert(42)$ the tag for the user name will
have an onmouseover-attribute that can execute harmful code.

Also sending users URLs can enable execution of harmful code and also
the AJAX calls can be exploited because they execute the code returned
(that should be clean JSON, but might not be).

The XSS attacks provides a quite dangerous set of attacks as they
potentially provide full control over what the client does from the
second it is executed.  The attack can basically just add a script tag
providing the full attack code.

\subsection*{Client-State manipulation}
Manipulating the client stage provides is a way to forge requests, the
typical usage would be either escalating the users privileges such as
in the Gruyere example or for manipulating sites to have a wrong
behavior, such as a shop thinking the total price of a shopping basket
it 0.

\subsection*{Cross-Site Request Forgery}
The XSRF attack is a bit similar to the XSS, as it also provides a way
to make the user execute unintended actions.  The Gruyere example is
deleting a snippet, but could also be to make users upload intended
things.  

An interesting detail in the type of attack is that it can for
instance be used in cases where you don't have admin rights, but need
the admins permissions to do other attacks such as XSS.  Also due to
the fact the the simplest way to perform an XSRF is by the get
parameters, a lot of developers is wrongly let to believe post is more
secure.  This however is not true as the attacking site could have a
hidden form that submits to the site under attack and then having a
JavaScript that automatically submits.

\subsection*{Path Traversal}
Path traversal is used to either reveal secret info or to
place/replace files in the system.  This can be anything from
replacing a list of users to replacing for instance the index-page of
a site or even worse in the cases where the uploads are not properly
checked even uploading files that will be executed by the server.
This can in worst case be used to replace the entire code base of a
site with the code intended by a malicious person.

\section*{XSS and XSRF in node.js}
% Re-implement the multiset datatype from week 1 using TypeScript. Make sure to take advantage of the new features in TypeScript.
The multiset in TypeScript is fairly easy.  One thing that could have
been done to make it resemble the Java version more is to use the
generics from TypeScript, which hasn't been done here.  Apart from
that it's quite straight forward using classes.

Two noteworthy things is that using classes makes the one-shot
closures becomes less natural (although still possible) for
encapsulating $trueRemove$ and the internal data types and that
TypeScript does not like the overload with two different return types.
This would probably be best handled by dividing it up in two functions
with different names.

\section*{Flapjax}
\begin{minted}[linenos, frame=lines]{js}
main() {
  print(plus(1,2));
}

plus(String s1, String s2) {
  return s1 + s2;
}
\end{minted}
The first program prints the result of the function 'plus' given two parameters. The function 'plus' requires two string parameters, but in this program, it is called with two int parameters. When the program is run in 'checked mode' we will get a type error because we try to call a function with the wrong type parameters. When the program is run in 'production mode' the program runs and prints '3'. This happens because the it ignores the string type of the parameter of the 'plus' function, and perform the plus on the ints 1 and 2 which results in 3.

\begin{minted}[linenos, frame=lines]{js}
main() {
  Object o1 = 1;
  Object o2 = 2;
  print(plus(o1,o2));
}

plus(String s1, String s2) {
  return s1 + s2;
}
\end{minted}
The second program is much the same as the first program. In this program though two objects o1 and o2 is initialized to 1 and 2 respectively in the main function. If the program is run in 'checked mode' as before we get a type error because we try to pass two objects as parameters to a function which takes two strings. If the program is run in 'production mode' again the program runs and prints '3'. This is for the same reason as the first program.\\

When runnig both programs in 'cheked mode' we got type errors. This is what one would expect since we try to pase parameters of the wrong type. In 'production mode' we got no warnings. Since 'checked mode' is used for developing and 'production mode' is used for launching the product it makes sence to get the warnings during development and the run with the "keep on truckin" philosophy after the product is launched.\\

Java is statically typed which means that all variables need to be declared before they can be used. This involves setting the variables type. The type determines the values a variable may contain and the operations that may be performed on it. Dart is dynamically typed and has optional types, which means that you can write programs with no types whatsoever. If you do choose to add types, this will not prevent the program from compiling and running, even if the types are incomplete or wrong.


\end{document}
